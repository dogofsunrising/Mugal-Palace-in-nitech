\documentclass[uplatex,dvipdfmx,a4paper]{jsarticle}

% ソースコードおよび結果の表示に関する設定:ここから
\usepackage{listings,jlisting,color}
\usepackage{plautopatch}
\usepackage{amsmath}
\usepackage[dvipdfmx]{graphicx}
\usepackage{diagbox}
\usepackage{hyperref}
\usepackage{xcolor}
\usepackage{tabularx}
\usepackage{titling}
\usepackage{float}




\lstset{
  aboveskip=20pt,
  belowskip=20pt,
}
\lstdefinestyle{source}{%
  language=matlab,
  basicstyle=\ttfamily\small,
  numbers=left,
  numberstyle=\ttfamily,
  xleftmargin=2em,
  basewidth=.51em,
  frame=shadowbox,
  rulesepcolor=\color{black},
}
\lstdefinestyle{result}{%
  basicstyle=\ttfamily\small,
  xleftmargin=2em,
  basewidth=.51em,
  frame=single,
  frameround=tttt,
}


\title{プログラミング応用レポート}
\author{チーム:ムガルパレス名工大店}
\date{2025年4月18日,2025年4月25日,2025年5月2日}
\setlength{\droptitle}{5em} % タイトルの位置を調整

\begin{document}
\maketitle

\section{はじめに}
本レポートは、2025年度名古屋工業大学プログラミング応用の課題レポートである。
本レポートでは、我々のチームが開発した「WizardSpellNameQuiz」の仕様の解説を行う。
本レポートは、以下の構成で進める。
\begin{itemize}
  \item 1章: はじめに
  \item 2章: プログラムの概要
  \item 3章: プログラムの実行結果
  \item 4章: おわりに
\end{itemize}
本プロダクトは、ハリーポッターapi\href{https://github.com/KostaSav/hp-api}{[githubのリンク]}を使用しており、ハリーポッターに登場する魔法の名前を当てるクイズゲームである。

\section{プログラムの概要}

プロダクトのアーキテクチャは、以下のようになっている。


\subsection{プログラムの目的}




\end{document}